\documentclass[12pt]{article}
\usepackage{amsmath}
%\usepackage{fullpage}
\usepackage[top=1in, bottom=1in, left=0.8in, right=1in]{geometry}
\usepackage{multicol}
%\usepackage{wrapfig}
\usepackage{enumerate}
\usepackage{amscd}
\usepackage{amssymb}
\setcounter{secnumdepth}{0}
\setlength{\columnsep}{0.1pc}
\setlength\parindent{0pt}
\setlength{\parskip}{\baselineskip}

\title{Information Retrieval}
\author{Lay Kuan Loh}
\date{\today}
\begin{document}

  \maketitle

  \vspace{-0.3in}
  \rule{\linewidth}{0.4pt}


% ------------------------------------------------------------------------------------- %
%                                                                                       %
%                                           QUESTION 1                                  %
%                                                                                       %
% ------------------------------------------------------------------------------------- %

\section{Writing numbers in variable byte notation}

Steps
\begin{itemize}
  \item Write a number in binary
  \item Pad it with zeros in front until the number of bits is a multiple of seven
  \item Chunk the bits into groups of seven
  \item First group has a ``1'' appended to it.

	 Other groups have a ``0'' appended to it. 
\end{itemize}

Example: Writing 2056 in variable byte notation
\begin{itemize}
  \item 2056 is 1000 0000 1000 in binary
  \item chunk it as 10000 0001000
  \item Pad it as 001 0000 000 1000
  \item Append the ``1''s and ``0''s to get: 1001 0000 0000 1000
\end{itemize}

Exercise: Write the following numbers in variable byte notation
\begin{enumerate}[(1)]
  \item Number: 1
	 \begin{itemize}
	   \item Binary: 1
	   \item Padded to have a multiple of seven bits: 000 0001
	   \item Append a ``1'' to the first group: 1000 0001
	 \end{itemize}
  \item Number: 1023
	 \begin{itemize}
	   \item Binary: 11 1111 1111
	   \item Padded to have a multiple of seven bits: 000 0111 111 1111
	   \item Append ``1''s and ``0''s to get: 0000 0111 0111 1111
	 \end{itemize}
  \item Number: 12345678
	 \begin{itemize}
	   \item Binary: 1011 1100 0110 0001 0100 1110
	   \item Padded to have a multiple of seven bits: 000 0101 111 0001 100 0010 100 1110
	   \item Append ``1''s and ``0''s to get: 1000 0101 0111 0001 0100 0010 0100 1110
	 \end{itemize}
\end{enumerate}


% ------------------------------------------------------------------------------------- %
%                                                                                       %
%                                           QUESTION 1                                  %
%                                                                                       %
% ------------------------------------------------------------------------------------- %

\section{Question}

What is the smallest number for which variable byte encoding is 
less efficient than standard four byte notation? 

In other words, for what number $n$ does the variable byte representation 
of $n$ require more bits than the standard four byte representation?

\underline{Answer:}

\begin{itemize}
  \item Idea: Initially, the numbers $1,2,3,\ldots$ are stored in variable byte notation
	 and do not take much space.
	 Slowly, as numbers increase, they take more space, until they start taking $\geq 32$
	 bit space, then variable byte encoding is less efficient,
	 as it chews up unnecessary bits by padding the representation.

  \item Variable byte notation: 

	 We have a number represented by 28 bits, then we pad it with four bits.	
	 Anything bigger, and standard four byte becomes better.
	 That is the binary number: 1111111 1111111 1111111 1111111	 

  \item Standard four byte notation: 32 bits needed
\end{itemize} 

% ------------------------------------------------------------------------------------- %
%                                                                                       %
%                                           QUESTION 1                                  %
%                                                                                       %
% ------------------------------------------------------------------------------------- %

\section{Question}


111001011110000010111111010110

How many bits would we have needed to encode this with the standard 4 byte integer encoding?

\underline{Answer}

\begin{itemize}
  \item Pad to a multiple of seven:
	 0000011 1001011 1100000 1011111 1010110
  \item Add bits in front:
	 10000011 01001011 01100000 01011111 01010110

	 5 bytes
\end{itemize}

% ------------------------------------------------------------------------------------- %
%                                                                                       %
%                                           QUESTION 1                                  %
%                                                                                       %
% ------------------------------------------------------------------------------------- %

\section{$\gamma$ Encoding}

$\gamma$ codes implement variable-length encoding by splitting the representation 
of a gap $G$ into a pair of length and offset. 
Offset is $G$ in binary, but with the leading 1 removed. 

For example, for 13 (binary 1101) offset is 101. 

Length encodes the length of offset in unary code. 

For 13, the length of offset is 3 bits, which is 1110 in unary. 

The  code of 13 is therefore 1110101, the concatenation of length 1110 and offset 101. 


\textit{Table 5.5}

\begin{tabular}{|c|c|c|c|c|}
  \hline
  \hline
  Number & Unary code & Length & Offset & $\gamma$ code \\
  \hline
  \hline
  0 & 0 & & &  \\
  1 & 10 & 0 & & 0 \\
  2 & 110 & 10 & 0 & 10, 0 \\
  3 & 1110 & 10 & 1 & 10, 1 \\
  4 & 1110 & 110 & 00 & 110, 00 \\
  9 & 11 1111 1110 & 1110 & 001 & 1110, 001 \\
  13 &  & 1110 & 101 & 1110, 101 \\
  24 &  & 1 1110 & 1000 & 11110, 1000 \\
  511 &  & 1 1111 1110 & 1111 1111 & 1 1111 1110, 1111 1111 \\
  1025 & & 111 1111 1110 & 00 0000 0001 & 111 1111 1110, 00 0000 0001 \\
  \hline
\end{tabular}

$\gamma$ codes are relatively inefficient for large numbers 
(e.g., 1025) as they encode the length of the offset in inefficient unary code. 
$\delta$ codes differ from  codes in that they encode the first part of the code (length) 
in  code instead of unary code. 

\subsection{Example: $\gamma$ code}

\begin{itemize}
  \item 13 in binary: 1101
  \item offset: 101 (leading one removed)
  \item length: length of offset in unary code

	 Length of offset for 13 is 3 bits, which is 1110 in unary
  \item $\gamma$ code of 13: 1110 101

	 Concatenation of length 1110 and offset 101
\end{itemize}

\subsection{Example: $\delta$ code}

\begin{itemize}
  \item Length is coded in $\gamma$ code, not unary code 
  \item $\delta$ code of 7 is 10 0 11
  \item 7 in binary: 111
  \item Offset of 7: 11 (leading one removed)
  \item Length: 2 bits

	 Encoded in $\gamma$ code as 

	 2 in binary: 10
	
	 Leading one removed: 0

	 1 bit in unary: 10

	 $\gamma$ code: 10 1
  \item In delta code: 100 11
\end{itemize}

% ------------------------------------------------------------------------------------- %
%                                                                                       %
%                                           QUESTION 1                                  %
%                                                                                       %
% ------------------------------------------------------------------------------------- %

\section{Question}

\begin{itemize}
  \item Compute the $\delta$ codes for the other numbers in Table 5.5. 
	 For what range of numbers is the $\delta$ code shorter than the $\gamma$ code? 

	 Example for 24: 

	 In binary: 1 1000

	 Offset: 1000

	 Length of offset: 4 bits

	 Write ``4'' in $\gamma$ code as: 110 00

	 Therefore 24 in $\delta$ code is: 11000, 1000
  \item $\gamma$ code beats variable byte code in Table 5.6 because the index contains stop 
	 words and thus many small gaps. 
 	 Show that variable byte code is more compact if larger gaps dominate. 

	 Length of $\gamma$ code for number $x$: $2\lfloor \log_2 x \rfloor + 1$

	 Length of $\delta$ code for number $x$: 
	 $\lfloor \log_2 x \rfloor + 
	 2 \lfloor \log_2(\lfloor \log_2 x \rfloor + 1)\rfloor + 1$

	 Larger gaps $\Rightarrow$ must write almost everything 

  \item Compare the compression ratios of code and variable byte code 
	 for a distribution of gaps dominated by large gaps.

	 Variable byte code: 

	 $8\lceil(\lfloor \log_2 x \rfloor)/7 \rceil$
\end{itemize}


\end{document}

% ---------------------------------------- END ---------------------------------------- %
