\documentclass[12pt]{article}
\usepackage{amsmath}
%\usepackage{fullpage}
\usepackage[top=1in, bottom=1in, left=0.8in, right=1in]{geometry}
\usepackage{multicol}
%\usepackage{wrapfig}
\usepackage{enumerate}
\usepackage{amscd}
\usepackage{amssymb}
\setcounter{secnumdepth}{0}
\setlength{\columnsep}{0.1pc}
\setlength\parindent{0pt}

\title{Lab 5 Probability}
\author{Lay Kuan Loh}
\date{\today}
\begin{document}

  \maketitle

  \vspace{-0.3in}
  \rule{\linewidth}{0.4pt}


% ------------------------------------------------------------------------------------- %
%                                                                                       %
%                                           QUESTION 1                                  %
%                                                                                       %
% ------------------------------------------------------------------------------------- %

\subsection{Yesterday's Puzzle}

Parity

$n$ different colors of hats
\begin{itemize}
  \item 
\end{itemize}


\subsection{Question 1}

\begin{enumerate}[(a)]
  \item Royal Flush: Dealt the 5 highest cards from a single suit.
	 
	 $|E| = 4$ as there are 4 suits, each with five highest cards

	 $\displaystyle\mathbb{P} = \frac{4}{\binom{52}{5}}$
  \item Four of a kind: Four cards of the same rank, one of another kind

	 $|E| = 13 \cdot \binom{4}{4} \cdot 48$

	 We have 13 possible ranks to choose for those four cards,
	 and there are 48 remaining cards to choose the other one from.

	 $\displaystyle\mathbb{P} = \frac{13 \cdot 48}{\binom{52}{5}}$
  \item Two Pairs: two pairs of same rank, one different

	 $|E| = 13 \binom{4}{2} \cdot 12 \binom{4}{2} \cdot 11 \binom{4}{1}$ 
  \item Flush:

	 All five cards of same suit, not in sequence

	 $|S| = 4 \binom{13}{5}$
  \item Straight flush:

	 Five cards in sequence, all of the same suit.

	 $|E| = 4 \cdot 10$

	 10 possible consecutive sequence. 
\end{enumerate}

% ------------------------------------------------------- %

\subsection{Question 2}

$\displaystyle \frac{\binom{n}{k}}{\binom{n}{k}!}$


% ------------------------------------------------------- %

\subsection{Question 3}

\begin{itemize}
  \item First $r$ bits have exactly $k$ zeros,
	 there other $M+N-r$ bits have exactly $M-k$ zeros.
  \item $\displaystyle \frac{\binom{r}{k} \binom{M+N-r}{r-k}} {\binom{M}{k}} $
\end{itemize}


% ------------------------------------------------------- %

\subsection{Question 4}

\begin{itemize}
  \item $J_I$ = \# jurors voting the defendant innocent.

	  $J_G$ = \# jurors voting the defendant guilty
  \item $A_I$ = Event the defendant is really innocent

	  $A_G$ = Event the defendant is really guilty
  \item $F_G$ = Event a juror votes guilty

	 $\mathbb{P}(F_G|A_I) = 0.1$

	 $\mathbb{P}(F_G|A_G) = 0.8$

  \item $F_I$ = Event a juror votes innocent

	 $\mathbb{P}(F_I|A_I) = 0.9$ 

	 $\mathbb{P}(F_I|A_G) = 0.2$
  \item $C$ = Event the jury makes the correct decision.
	 \begin{eqnarray*}
	    C &=& \mathbb{P}(J_G \geq 9|A_G)\mathbb{P}(A_G) 
		    + \mathbb{P}(J_I \geq 9|A_I)\mathbb{P}(A_I) \\
	      &=& (0.65)\sum_{i=9}^{12}\binom{12}{i}0.8^i 0.2^{12-i}
		    + (0.35)\sum_{i=9}^{12}\binom{12}{i}0.9^i 0.1^{12-i} \\
	      &=& 0.857
	 \end{eqnarray*}
  \item \begin{eqnarray*}
	   \text{Percentage of defendants found guilty by the jury} \\
		      = \mathbb{P}(J_G \geq 9) \\
	  		=\mathbb{P}(J_G \geq 9|A_G)\mathbb{P}(A_G)
			   +\mathbb{P}(J_G \geq 9|A_I)\mathbb{P}(A_I) \\
		      = (0.65)\sum_{i=9}^{12}0.8^i 0.2^{12-i}
			   +(0.35)\sum_{i=9}^{12}0.9^i 0.1^{12-i} \\
	\end{eqnarray*}
\end{itemize}



% ------------------------------------------------------- %

\subsection{Question 5}

\begin{enumerate}[(a)]
  \item William's parents must be Bb since otherwise his sister
	 cannot be bb.
	 Therefore William can be Bb, Bb, or BB.
	 So he has probability $\frac{2}{3}$ to have a blue-eyed gene.
  \item If William is $Bb$ and wife is bb, his first child has 
	 blue eyes with probability $\frac{1}{2}$. 

	 If William is $BB$ and wife is bb, no chance his first child has blue eyes.

	 Therefore $\mathbb{P}(\text{first child has blue eyes is})
	 = \frac{2}{3} \cdot \frac{1}{2} + \frac{1}{3} \cdot 0 = \frac{1}{3}$
  \item W-Bb: William has Bb genotype

	 W-BB: William has BB genotype

	 CBrown: William's first child is brown

	 \begin{eqnarray*}
	   \mathbb{P}(W-Bb|CBrown)
	 	&=& \frac{\mathbb{P}(CBrown \cap W-Bb)}{\mathbb{P}(CBrown)} \\
	 	&=& \frac{\mathbb{P}(CBrown|W-Bb)\mathbb{P}(W-Bb)}
			{\mathbb{P}(CBrown|W-Bb)\mathbb{P}(W-Bb)
			+\mathbb{P}(CBrown|W-BB))\mathbb{P}(W-BB)} \\
	 	&=& \frac{\frac{1}{2} \cdot \frac{2}{3}} 
			{\frac{1}{2}\cdot \frac{2}{3}+ 1\cdot\frac{1}{3}} \\
		&=& \frac{1}{2}\\
	 \end{eqnarray*}

	 $\mathbb{P}(SCBrown) = \mathbb{P}(SCBrown|W-Bb)+\mathbb{P}(SCBrown|W-BB)
				= \frac{1}{2} \cdot \frac{1}{2} + 1\frac{1}{2}
				= \frac{3}{4}$
\end{enumerate}













\end{document}

% ---------------------------------------- END ---------------------------------------- %
