
\section{Lecture: July 16, 2013}

\subsection{Your website}

Course website: \texttt{www.epgy.sherolchen.com}

Computers
\begin{itemize}
  \item To reduce amount of human work needed
  \item Cloud computing
        \begin{itemize}
          \item Share data online
	   \item Google has a computer without a hard drive because of this
        \end{itemize}
\end{itemize}

Domain names:
\begin{itemize}
  \item Maybe \$10 / year to buy one
  \item \texttt{wordpress.com} is free
  \item Important
        \begin{itemize}
          \item Allows people to see what you have done
	   \item Employers always google you to see what pops up these days 
        \end{itemize}
  \item Have an online resume
        \begin{itemize}
          \item Easy to edit when you need it
	   \item Takes a while for google to list your new CV when you need it
        \end{itemize}
\end{itemize}

Stay up to date with the news
\begin{itemize}
  \item Java has lots of security issues
        \begin{itemize} 
          \item One day you may need to know that
        \end{itemize}
  \item SDK/API changes all the time for iPhone
        \begin{itemize}
          \item Software development kit
	   \item Need to port things from older to newer all the time
        \end{itemize}
\end{itemize}

Building your resume
\begin{itemize}
  \item Track what you are doing over the years/ in the past
        \begin{itemize}
          \item Useful for filling out applications for internships/ etc
        \end{itemize}
 \item Link your projects! - Important
 \item Don't choose stupid domain names/ page designs, etc
 \item Open processing - \texttt{http://www.openprocessing.org}
        \begin{itemize}
          \item like Java
	   \item Good for rapid processing
	   \item Convenient to display your projects
        \end{itemize}
\end{itemize}

\texttt{TheGreatFirewallOfChina.com}
\begin{itemize}
  \item What websites are blocked in China?
  \item Twitter
  \item Facebook 
\end{itemize}

App Development
\begin{itemize}
  \item Graphics are important!
\end{itemize}

AI 
\begin{itemize}
  \item Very interdisciplinary 
  \item Finding patterns in big data 
\end{itemize}

Google drive
\begin{itemize}
  \item Can do most Microsoft word-ish stuff online
  \item Stuff online is good - easy to update stuff quickly
\end{itemize}

How websites work
\begin{itemize}
  \item Click on link
  \item Browser downloads info from website into your computer
  \item Browser converts the HTML to code you can read
  \item Therefore don't click on malicious links to write stuff to your computer
\end{itemize}

\subsection{What is Artificial Intelligence?}

1956 - AI term coined 
\begin{itemize}
  \item Dartmouth conference
\end{itemize}

Strong vs Weak AI
\begin{itemize}
  \item John Searle
        \begin{itemize}
          \item Strong: ``A physical symbol system can have a mind and mental status''
		
		  An intrinsic part of what you created
	   \item Weak: ``A physical symbol system can act intelligently.''
        \end{itemize}
  \item Does it matter? 
\end{itemize}

Approaches to AI
\begin{itemize}
  \item Systems that think? act? link humans
  \item Systems that think? act? rationally
  \item Textbook: Russell/Norvig, 2

	 Needs updating! 
\end{itemize}

Human Simulation
\begin{itemize}
  \item Systems that think like humans
  \item Cognitive Sciences - study of human and brain behavior
  \item Cybernetics
  \item Asimov's positronic brain
        \begin{itemize}
          \item Centrol computer for robot	
	   \item Provide it with consciousness recognizable to humans 
        \end{itemize}
  \item Ray Kurzweil: Technological Singularity
        \begin{itemize}
          \item Can machines get smarter by themselves
	   \item Can machines get so smart humans can't understand what they do?
        \end{itemize}
  \item Early neural networks 
  \item How do brains function? 
\end{itemize}

Logical Machines
\begin{itemize}
  \item Systems that think rationally
  \item Theorem provers
  \item Logic-based systems

	 Take the human out of it
\end{itemize}

Approaches to AI - Turing Test
\begin{itemize}
  \item ``Systems that act like humans''
  \item The imitation game
        \begin{itemize}
          \item Computer and Humans talking
	   \item Is it a computer or human? 
	   \item What background knowledge should a human know?
        \end{itemize}
  \item Siri
        \begin{itemize}
          \item Intelligent PA developed by Apple 
        \end{itemize}
  \item Joseph Weizenbaum
        \begin{itemize}
          \item ELIZA
	   \item chatbot
	   \item Human? Intelligent? Psychotherapist?
        \end{itemize}
  \item Can you fool 70\% of people into thinking your program is a human? 
  \item Loebner Test
  \item Internet dating sites?

	 Computer girlfriend/boyfriend 
\end{itemize}

Humans are better than computers at reading weirdly scrambled words
\begin{itemize} 
  \item ``It you don't baliefve me rifd zis yorgsehf.''
\end{itemize}

Rational Agent
\begin{itemize}
  \item Search
  \item Machine Learning
  \item Probabilistic models
  \item Constraint satisfaction
  \item Perception/ Sensors
  \item Robotics
\end{itemize}

John McCarthy and Lisp (1958)
\begin{itemize}
  \item Problems with AI
        \begin{itemize}
          \item Algorithmic complexity
	   \item Logical Systems and Neural Nets inappropriate
        \end{itemize}
  \item Reframing of AI - Rational agent model
  \item Expert Systems and Knowledge (1970s)
  \item Natural Knowledge 
\end{itemize}

Artificial Flight
\begin{itemize}
  \item Natural Flight: Birds
  \item Artificial Flight: 

	 Gliding airplanes 

	 Looked into aerodynamics
  \item Same idea for AI 
  \item Not necessary to create the human being

	 Not cognitive science approach
  \item Work towards desired OUTCOME instead
\end{itemize}

Recent progress
\begin{itemize}
  \item Deep learning
  \item Real tools
  \item Applications 
\end{itemize}

Big Data
\begin{itemize}
  \item How to mine big data
  \item Learn from the right parts of information?
  \item How do we store purchasing history of Amazon.com
\end{itemize}




