% ---------------------------------------------------------- %
%                        DOCUMENT TYPE                       %
% ---------------------------------------------------------- %

\documentclass[12pt]{article}

% ---------------------------------------------------------- %
%                        PREAMBLE                            %
% ---------------------------------------------------------- %
\usepackage{amsmath}
%\usepackage{fullpage}
\usepackage[top=1in, bottom=1in, left=0.8in, right=1in]{geometry}
\usepackage{multicol}
%usepackage{wrapfig}
\usepackage{enumerate}
%\usepackage{permute}
%\setcounter{secnumdepth}{0}
\setlength\parindent{0pt}
\setlength{\columnsep}{0.1pc}
\setlength{\parskip}{5pt}


% ---------------------------------------------------------- %
%                          DOCUMENT                          %
% ---------------------------------------------------------- %

\begin{document}

% ---------------------------------------------------------- %
%                                TITLE                       %
% ---------------------------------------------------------- %

\title{Undergraduate Admissions Talk \\
	by a Stanford Admissions Officer for EPGY students}
\author{Lay Kuan Loh}
\date{July 29, 2013}
\maketitle

% ************************************************************** %

\textbf{Disclaimer:}
\textit{This was written from memory after hearing a talk for EPGY
High School students at Stanford University on July 29, 2013. 
I am a counselor for EPGY's High School program at the point of writing this.
Although the college applications process has passed for me,
I think some people will find what I heard useful.
Do not take everything said here at face value, as I unfortunately
do not have an eidetic memory. Check the Stanford website
or call them up to verify what I have written here.} 


% ************************************************************** %

\section{The High School Transcript}

Freshman year scores are NOT considered AT ALL; however it is unlikely
in many cases that students who did really badly in freshman year show
tremendous improvement in their later years. 
If that did happen to you, it would be a good thing to highlight however. 

The most important thing in the application process is the high
school transcript. Admissions officers want to look for evidence
that you have challenged yourself as far as possible with what your
high school offers and done well in those courses. 

Notice the words ``as far as possible''. The admissions officers
evaluate applicants in the context of their high school. Lets say Alice who is from
a rich family who sends her to an expensive, top notch high school 
with multiply chances for honors and AP classes and she takes them.
Bob however is from a poorer family and neighborhood, which does not
have as many honors classes for offer. However, lets say that Bob's
High School gives students chances to learn other useful skills
like how to fix cars, and Bob takes advantage of that and works part time
using that knowledge. The Bob is not going to be viewed as a ``worse''
student than Alice just because he has fewer AP classes.
Not everyones' school has the same resource, and admissions officers
take this fact seriously.

Similarly, grades are taken in the context of other students.
Each student's high school counselor is asked to submit a profile of the school,
including stuff like how many students take AP classes, average scores,
the curriculum taught in school, etc, so that an applicant can be judged
in the context of their high school. This was stressed again and again,
as what is considered as ``good'' grades in one school nay be considered average
in another. 

It is never about just grades and the transcript alone.
The admissions officers estimate that at least 80\% of all Stanford applicants
are academically qualified for Stanford. 
To be admitted, Stanford applicants must have something else that makes them
stand out, be it extra-curricular activities, 
and interested job (paid or not), or otherwise an interesting experience
they have lived through and learned something from. 

The High School Transcript is much more important than
your standardized test scores, as fours years of school is more
important than four hours of testing to the admissions officers.
Which brings us to $\ldots$

% ************************************************************** %

\section{Standardized Testing}

You must take the SAT or ACT. The best score only is considered,
though you are asked to report all scores.
It is recommended you do both if you have the time and inclination,
as some people test better in one than another.
Your scores for each section are considered separately.
The averages of standardized tests for admitted applicants is not published,
but the admissions officer said it was ``above 40''. 

You must choose at least two subject tests of the SAT to take,
which should naturally be the two subject tests you can score best in. 

% ************************************************************** %

\section{Intellectual Vitality}

Stanford values intellectual vitality, and the best way to show
this is to take the hardest classes you can do well in,
and doing stuff in the summer holidays.

For instance, the admissions officer explains, doing EPGY is a 
good way to show you enjoy taking challenging courses,
as EPGY classes cull material taken from Stanford college classes.
As a counselor for Artificial Intelligence, let me state
that I find some of the stuff they are doing challenging as well.
I have to do all the assignments before class myself, or I would have 
trouble helping the students during the study session.

Other things showing intellectual vitality are any other
extra-curriculars you may have done. 

% ************************************************************** %

\section{Letters of recommendation}

Two compulsory letters from a high school teacher who taught you
in the 11th or 12th grade in the following fields of study:
\begin{itemize}
  \item English
  \item Math
  \item Sciences
  \item Social Sciences
  \item Foreign Language
\end{itemize}

One optional, third letter from any other person who is not a relative,
who knows you well.
This person can be a coach, church leader, or boss - someone you met in your
extra curricular's. Or this person can be a high school
teacher who taught a class not from the five core areas listed above,
such as journalism, or a teacher that taught you in 10th grade.
You may not submit more than three letters of recommendation.  

For all these letters, chose someone who knows you well and can write
a compelling case for you. The recommendation letter is extremely important,
and a teacher that vouches enthusiastically for you will make a big difference. 

% ************************************************************** %

\section{Extra-curriculars}

Besides good grades, this is what will make the difference between admission
and rejection. 
Stanford values students who have put in the time to something they are
particularly passionate about, be it a subject in school or a sport.

One thing the admissions officer stressed is not to go on a club binge
in junior by joining every club that exists in your high school,
and putting little time into it. You should not try to
fill every line on the ``extra-curriculars'' sheet in the application.
Instead, devote a lot of time to one thing you can excel in, and that
will speak a lot more. 

Jobs, paid or otherwise, count as extra-curriculars. Church activities,
and watching your baby sister at home also counts if you have learnt
something about it. 

% ************************************************************** %

\section{Homeschooled Applicants}

The High School Transcript is the most important thing for applicants
who went to regular high schools, private or public.
However, standardized testing becomes much more important for homeschooled
applicants, as they are often the only thing the admissions officers have
to evaluate them on against the rest of the applicant pool.
It is vital that the homeschooled applicant give a very clear idea of what
they studied. 
If homeschooled applicants have done other programs like EPGY,
they should submit that information as well. 

% ************************************************************** %

\section{Legacy Applicants}

Stanford defines ``legacy applicants''
as a student whose parent or stepparent attended Stanford.
Siblings, aunts, uncles, or grandparents do NOT count. 
Legacy applicants are treated differently from other applicants,
but being a legacy applicant alone will never get an applicant through the door. 

% ************************************************************** %

\section{Donations}

This was not discussed, but I can't resist adding it in.

If you are loaded and can donate at least like \$1 000 000,
you may most likely get it as long as the admissions officers
feel you are able to successfully graduate from Stanford. 
Call the school for more information on this though, don't 
take me at my word. 

% ************************************************************** %

\section{Financial Aid}

Stanford is need-blind for U.S. Citizens, Green Card holders,
and residents. Stanford does not offer scholarships.
Aid is determined solely on an applicant's ability to pay. 

Stanford is need-aware for students who are not 
U.S. Citizens or Green Card holders,
from a high school with a foreign Zip Code. 

Stanford hopes to become need-blind to all applicants
in about 4-5 years. 

% ************************************************************** %

\section{The Common Application}

The Common Application this year becomes live on August 1, 2013.
This is about the fourth version so far. 
Information you submit to the common application goes to all
colleges you apply for, so write about something useful to
colleges to evaluate you.  

% ************************************************************** %

\section{The Stanford Supplement}

\subsection{Essay 1: Tell us about something you found intellectually stimulating}

You can write about a subject in school that particularly interested you.
One person wrote about watching jeopardy with her grandparents though,
and was successful in her application. 

\subsection{Essay 2: Write a letter to your future roommate at Stanford}

Note that this letter is never actually sent to a successful applicant's
future roommate at Stanford. Do not take it too seriously,
and write about what stuff you want to start sharing, or plan the future
room cleaning schedule. 
Use this to allow a stranger to get to know you better. 


\subsection{Essay 3: What is important to you and why?}

This question was once: ``What about Stanford makes it a good fit for you?''
or something along this lines.
The admissions office changed the question to ``reduce stress on applicants'',
though they are unsure if they have achieved their goal. 

Many applicants answered about ``What is important to me'', but
it is important to remember to explain ``why'',
as this is the part that tells the admissions officers about you. 

\subsection{Short Answer Questions}

\begin{enumerate}[(1)]
  \item \textit{Five words to describe you:}

	Most people wrote something like:
	``Hardworking, determined, compassionate, friendly, outgoing''
	
	Someone wrote: ``Five words cannot describe me''
  \item \textit{One historical event you would like to witness:}

	The most common reply is: ``Neil Armstrong landing on the moon.''
	Once, a careless applicant wrote: ``Lance Armstrong landing on the moon.''
\end{enumerate}

\subsection{About writing essays}

You should absolutely review the essays with a college counselor or a
trusted friend with knowledge about the applications,
to make sure they flow well and you don't make silly mistakes. 

% ************************************************************** %

\section{College Athletics}

Check the Stanford Athletics website for listings of coaches and
get in touch with them. 
All applications will still pass through the main Admissions office.

% ************************************************************** %

\section{The Arts Supplement}

Students who want to submit art materials for consideration
may submit materials from these sections:
\begin{enumerate}[(1)]
  \item Playing music Instruments
  \item Theatre
  \item Drawing
\end{enumerate}
These materials are sent to faculty to review. 

% ************************************************************** %

\section{Should I apply for Restrictive Early Action (REA)?}

The only restriction on applying for REA is that you are not allowed
to apply to another private American college early.
You cannot apply to Harvard and Stanford early.
You are still allowed to apply for state colleges and U.S. Military
Academies early.

The admissions officer stressed that you should only apply early
if your application is as good as it will get: you are not expected to
improve your application in that extra two months until the regular
decision deadline.
The only advantage to applying early is that you will be notified of the
decision earlier, so that may allow for better planning.

While the admit rate for REA is higher than regular decision,
this is because the REA pool applicants are generally better than
the regular decision applicants, and tend to be the focused
applicants who do not put off things. 
The admissions officer said that students admitted in REA would have been
admitted in Regular decision, and vice versa.

The decisions for REA are mailed on December 15, so that takes them 6 weeks
to review all of them.
The regular decision takes longer to review as the applicant pool is much larger.
All admitted applicants have until May 1 to decide on whether they will attend. 

% ************************************************************** %

\section{Take-Away Points}

One thing the admissions officer really stressed is that
Stanford aims to evaluate each applicant as a person,
not as their test scores and lines on a page.
Successful applicants are able to present themselves as a whole person to the
admissions officers.

If you are aiming for Stanford, the most important thing is to 
find something that interests you and distinguish yourself as 
much as possible in it, and take challenging classes you can do well in. 

Finally, he reminds the EPGY students that no matter where they go,
they will be a success. College admissions are often luck of the draw.
Maybe you were born to good parents who gave you every advantage in life.
Maybe your parents were never too interested in you,
and you had to fight harder than others for everything you achieved in life.
I personally do not believe that people are created equal,
and certainly some people are born with more than others.
``The universe is fair because it is unfair to everyone.'' I'm not sure
who said it, but it rings very true in my mind. 

% ---------------------------------------------------------- %
%                          END DOCUMENT                      %
% ---------------------------------------------------------- %

\end{document}


